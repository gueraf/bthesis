\chapter{Steady State Analysis of the Gray-Scott Reaction System}
\label{ch:app1}
\paragraph{Calculation of the steady states}
The steady state condition prescribes 
\begin{align}
\label{app:1}
0 \overset{!}{=} \frac{du}{dt} &= -\rho u v^2 + F(1-u) = f_1(u,v)\\
\label{app:2}
0 \overset{!}{=} \frac{dv}{dt} &= \rho u v^2 - (F + \kappa) v = f_2(u,v)
\end{align}
Combining \eqref{app:1} and \eqref{app:2} leads to
\begin{align}
\label{app:3}
u = 1 - \frac{F + \kappa}{F}
\end{align}
Plugging \eqref{app:3} into \eqref{app:2} gives
\begin{align}
- \frac{F + \kappa}{F} v(v^2 + \rho v - (F + \kappa)) = 0
\end{align}
The trivial steady state is
\begin{align}
\vec{x}_{s1} = 
\begin{bmatrix}
1.0 & 0.0
\end{bmatrix}^T
\end{align}
The two other steady states are
\begin{align}
\vec{x}_{s2/3} = 
\begin{bmatrix}
\frac{1}{2} (1 \pm \sqrt{1 - 4 \frac{(F+\kappa)^2}{F \rho}}) &
\frac{F}{2 (F + \kappa)} (1 \mp \sqrt{1 - 4 \frac{(F + \kappa)^2}{F \rho}})
\end{bmatrix}
\end{align}
For the given parameters ($F = 0.04$, $\kappa = 0.06$ and $\rho = 1.0$) one has
\begin{align}
\vec{x}_{s2} = \vec{x}_{s3} = 
\begin{bmatrix}
0.5 & 0.2
\end{bmatrix}^T
\end{align}

\paragraph{Classification of Eigenvalues}
The Jacobian of the system is
\begin{align}
J = \begin{bmatrix}
\frac{\partial f_1}{\partial u} & \frac{\partial f_1}{\partial v} \\[0.3em]
\frac{\partial f_2}{\partial u} & \frac{\partial f_2}{\partial v}
\end{bmatrix}
=
\begin{bmatrix}
-\rho v^2 - F & -2 \rho u v \\[0.3em]
\rho v^2 & 2 \rho u v - (F + \kappa)
\end{bmatrix}
\end{align}
For the given parameters and at the steady state $\vec{x}_{s1}$ this is
\begin{align}
J_{s1} = \begin{bmatrix}
-0.04 & 0 \\
0 & -0.1
\end{bmatrix}
\end{align}
The eigenvalues of this matrix are $\lambda_{11} = -0.1$ and $\lambda_{12} = -0.04$. Since $\Re(\lambda_{11}) < 0$ and $\Re(\lambda_{12}) < 0$ this steady state is stable. \par
For the given parameters and the steady states $\vec{x}_{s2} = \vec{x}_{s3}$ this is
\begin{align}
J_{s2} = J_{s3} = 
\begin{bmatrix}
-0.08 & -0.2 \\
0.04 & 0.1
\end{bmatrix}
\end{align}
The eigenvalues of this matrix are the roots of the characteristic polynomial
\begin{align}
(-0.08-\lambda)(0.1-\lambda)+0.08=\lambda (\lambda - 0.02)
\end{align}
One obtains $\lambda_{21} = \lambda_{31} = 0$ and $\lambda_{22} = \lambda_{32} = 0.02$.
%
%
%
%
%