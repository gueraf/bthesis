\section{Illustrative Model: The Gray-Scott System}
Throughout this thesis, the Gray-Scott reaction system described in \cite{pearson_complex_1993} will be used as a primary example to illustrate the presented techniques, algorithms and results. 

The chemical reaction system consists of two species U and V subject to the reactions as described in the following: \\
\begin{align}
\label{eq:r1}
\cee{\emptyset{} &->[F] U} \\
\label{eq:r2}
\cee{U &->[F] \emptyset} \\
\label{eq:r3}
\cee{V &->[F+\kappa] \emptyset} \\
\label{eq:r4}
\cee{U + 2V &->[\rho] 3V}
\end{align}
with positive reaction rate constants $F$, $\kappa$ and $\rho$.

The two species U and V are also subject to diffusion. Approaches towards modelling this spatial dependency of the system will be discussed in chapters (ref) and (ref) for the deterministic and the stochastic case. 

The reactions given above can be interpreted as follows: Reactions \eqref{eq:r1} and \eqref{eq:r2} describe degradation and production of U particles, respectively. Since they are the opposite to each other, the process of production and degradation of U is reversible. Equation \eqref{eq:r3} describes degradation of V particles. Note that there is no direct production of V in the system. Finally, reaction \eqref{eq:r4} describes the autocatalytic\footnote{A reaction is said to be autocatalytic if it is catalyzed by its products \cite[pg.\ 907]{atkins_physical_2009}. In this example, V particles can only be generated if there are already some available in the system to catalyze the reaction.} conversion of U particles to V particles. The symbol $\emptyset$ represents species that are available in excess (i.e.\ not rate-limiting), but are of no further interest for the modeled process. 

The Gray-Scott model is of great interest since it
\begin{itemize}
\item can be used to describe real-world phenomena.
\item is subject to pattern formation (for certain parametrizations).
\item is computationally expensive due to spatial dependencies.
\end{itemize}
All these properties will be examined more closely throughout the thesis. 
%
%
%
%
%
%
%
%
\ifdebug

Reasoning: Molecules must collide before they can react (for 2nd and higher order). $\rightarrow$ reaction within compartment makes sure that molecules are close to each other (have a chance to collide)

\underline{Reactions}
\begin{align}
\label{eq:r1}
\cee{\emptyset{} &->[F] U_{i,j,k}} \\
\label{eq:r2}
\cee{U_{i,j,k} &->[F] \emptyset} \\
\label{eq:r3}
\cee{V_{i,j,k} &->[F+\kappa] \emptyset} \\
\label{eq:r4}
\cee{U_{i,j,k} + 2V_{i,j,k} &->[\rho] 3V_{i,j,k}}
\end{align}
with positive reaction rate constants $F$, $\kappa$ and $\rho$.

\underline{Diffusion}
\begin{align}
\label{eq:d1}
\cee{S_{i,j,k} &<->[d_S] S_{i\pm{}1,j,k}} \\
\cee{S_{i,j,k} &<->[d_S] S_{i,j\pm{}1,k}} \\
\cee{S_{i,j,k} &<->[d_S] S_{i,j,k\pm{}1}}
\label{eq:d3}
\end{align}
for the two species $\text{S} \in \{\text{U, V}\}$ with positive diffusion constants $d_\text{U}$ and $d_\text{V}$. 

The spatial dependency of the system is reflected by the compartment index (i,j,k). In a uniform, three-dimensional grid with spacing $h$, compartment $C_{i,j,k}$ is equal to the cube $\left[(i-1)h,ih\right] \times \left[(j-1)h,jh\right] \times \left[(k-1)h,kh\right]$. It is assumed that only particles close to each other (i.e.\ within a compartment) can  react according to equations \eqref{eq:r1} - \eqref{eq:r4}. Equations \eqref{eq:d1} - \eqref{eq:d3}, on the other hand, describe how particles can migrate from one compartment to another. Diffusion is therefore modeled as a jump process between neighboring compartments. Compartment-based simulation of diffusion as well as its relation to Fick's law of diffusion will be introduced in more detail in chapter \eqref{stochdiff}. 
\fi